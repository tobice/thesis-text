\chapter*{Introduction}
\addcontentsline{toc}{chapter}{Introduction}

The Internet of today is full of data and the amount of the data is constantly increasing. By data we mean literary everything that can be found online, may it a huge source of information like Wikipedia or a small Excel table produced by a local government office. Unfortunately, our current ability to use the whole potential of all currently available data is very limited. We want to to explore the data, we want to connect and combine different data sets, we want to look for patterns in the data, we want to make the data available to end users. All of that is possible, yet usually time and resources demanding. 

Let us consider for example the Excel table.  XLSX is a proprietary format that can be officially opened only by a paid proprietary software that offers a large yet still limited set of tools that allow the user to work with data inside the Excel table. It may very well happen that the proprietary tool will not allow us to do what we want at one moment. We might want to combine the contained data with another data set in a completely different format. Or we might want to create a new specialized interactive visualization of the data and publish it online. The latter one is  a pretty reasonable and imaginable request yet to get it done, we would most likely need a computer specialist or possibly even a developer. Those are rare and expensive.

The answer are application generators. Those are tools that allow non-developers to do developers' work which in this case means generate an interactive online application based on the given data set. Such generators already exist, for example for the aforementioned XLSX format (or in general tabular data) but they are usually very limited, in this case to a single type of data.

One of the recent positive trends is that more and more information is published using the Resource Description Framework \cite{rdf} while utilizing the Linked Data model \cite{ld}. Besides being open and offering a natural way of linking and combining different data sets, this framework is able to express any kind of information by extending the raw data with machine-readable semantic meta-data. So whereas tabular based application generator is able to work with only a single type of data, Linked Data based application generator would have no such limit. As such a generator could literary understand the data, it would allow the user to generate an application tailored for this specific type of data.

The main goal of this thesis will be to create such a Linked Data driven application generator.

The main focus will be on allowing the non-developers do developers' work. Obviously, that is possible only to a certain extent. The data represented in RDF can be understood with software, but first someone (typically the developer) has to teach the software how to understand it. For example, for data containing geospatial information (RDF gives a way how to describe such information), the developer would have to extend the generator with a plugin showing such data on a~map.

\section*{Motivation}

Let us describe more in depth how our application generator will work and how it will help its users. To make it more organized, we will separate the descriptions by different type of users that should benefit from our generator. For each user, we will describe his role, his motivation, use case typical for this user and eventually the contribution of this generator, i. e., how this generator is going to help him.

\begin{itemize}
\item \textbf{Developer}

\textit{Role description:} A software engineer with deep technical knowledge.

\textit{Motivation/Use case:} A developer is given a task at his job to create an interactive online map application that visualizes provided company geospatial data. The developer is smart and wants to avoid doing the same thing again the next time. Therefore, he would prefer to create a universal solution that would allow other non-developer employees to create the applications on their own and do it with any data containing geospatial information. Also, it is very likely that he will be given similar task in the near future, but with a different data type (e.g. some special graphs) and the developer would like to re-use as much work as possible from the first time.

\textit{Contribution:} Thanks to our application generator, the developer will have to focus only on developing the actual visualizer plugin. The generator will work both as a framework and a platform, solving many problems up front for the developer. As as a platform, the generator will handle user agenda, registration of new plugins, application management (creating, updating, deleting, publishing), common application configuration and so on. As a framework, the generator will define a standard plugin interface which will allow seamless integration. Plus it will provide many drop-in solutions for typical challenges that the developer might encounter across multiple plugins (e.g. caching, multi-language support, support for fixing missing or incorrect data labels, re-usable user interface components etc.).

\item \textbf{Data Analyst}

\textit{Role description:} A data analyst is not necessarily a developer, but he has enough knowledge to work with data on a low-level, typically convert data to RDF format or to augment existing RDF data sets with different meta-data. 

\textit{Motivation/Use case:} Let us say that a data analyst works at the same company as our developer. He has a couple of interesting data sets that he would like to make available for other employees so that they could use them to generate applications with the plugin developed by the developer.

\textit{Contribution:} By implementing the interface defined by the framework, each visualizer plugin clearly describes what the data should look like. Therefore 
\end{itemize}


Let us sum it up. Our application generator will be:

\begin{enumerate}
\item A framework
\item A platform
\end{enumerate}
