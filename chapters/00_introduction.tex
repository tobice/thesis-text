\chapter*{Introduction}
\addcontentsline{toc}{chapter}{Introduction}

The Internet of today is full of data and the amount is constantly increasing. By data we mean literary anything that can be found online, may it be a huge source of organized information like Wikipedia or a small table created with Microsoft Excel by a local government office. Unfortunately, our ability to use the whole potential of all currently available data is very limited. We want to to explore the data, we want to connect and combine different data sets, we want to look for patterns in the data, we want to make the data available to end users. All of that is possible, yet usually time and resources demanding. 

Let us consider for example the aforementioned table produced by Microsoft Excel.  XLSX is a proprietary format that can be officially opened only by a paid proprietary software. This software offers a large yet still limited set of tools. It may very well happen that it will not allow us to do what we want. Given the topic of this thesis, let us say that we want to create from the table a new specialized interactive visualization (or an \textit{application}) and publish it online. To get this done,  we would most likely need a computer specialist or  even a developer. Those people are relatively rare and expensive.

Tools that attempt to solve this particular problem are called \emph{application generators}. They allow non-developers to do developers' work which in this case means to generate from the data an interactive (possibly online) application. Such generators already exist, for example for the aforementioned XLSX format (or in general for tabular data) but they are usually very limited. One reason for this limitation is that the tabular data carry very little semantic information with them. One row represents an entity and each column contains one entity attribute. The first row might (or might not) contain labels for the attributes. That is all that the tabular data based generator can work with.

One of the positive recent trends is that more and more information is published using the Resource Description Framework \cite{rdf} while utilizing the Linked Data model \cite{ld}. Besides being open and offering a natural way of linking and combining different data sets, this framework is able to express any kind of information by extending the raw data with machine-readable semantic meta-data. In other words, data represented in RDF are, unlike the tabular data, not limited in what semantic information they can carry with them. A \emph{Linked Data driven application generator} could benefit significantly from this feature. As such a generator could literary \emph{understand} the data, it would allow the user to generate way more versatile applications, tailored for all different kinds of data. 

The main goal of this thesis will be to create such a \emph{Linked Data driven application generator}. That means, as already suggested, that we will be focusing on allowing the non-developers do developers' work. Obviously, that is possible only to a certain extent. There already exist applications based on Linked Data. One example could be Justinian \cite{justinian}, online tool for browsing laws and other legal documents. This tool is very domain specific and quite complex. One could hardly imagine that a non developer could create Justinian just by himself. Nevertheless, what one could imagine is that a developer could extend the \emph{application generator} with a plugin with capabilities similar to Justinian. Non-developers would be then able to spawn their own \emph{Justinians} but with different data sets (e.g. laws from different countries).

\section*{Motivation and contribution}
% * <tobiaspotocek@gmail.com> 2016-06-18T23:57:34.671Z:
%
% > Motivation and contribution}
%
% Ohledně této sekce si nejsem jistý. Psal jsem ji se záměrem vysvětlit čtenářovi konkrétně co bude výstupem práce, nicméně ten popis zachází do značných detailů a většina z toho je pak zanalyzována a vysvětlena až někde později. Zpětně by tato sekce mohla být možná zcela vynechána.
%
% ^.
Let us describe more in depth how our \emph{application generator} will work and how it will help its users. To make it more organized, we will separate the descriptions by different type of users that should benefit from our generator. For each type of user, we will describe his role, his motivation (or perhaps a use case typical for this user) and eventually the contribution of this generator, i. e., how this generator is going to help him.

\begin{itemize}
\item \textbf{Developer}

\textit{Description:} A software engineer with deep technical knowledge.

\textit{Motivation/Use case:} A developer is given a task at his job to create an interactive online map application that visualizes provided company geospatial data. The developer is smart and wants to avoid doing the same thing again the next time. Therefore, he would prefer to create a universal solution that would allow other non-developer employees to create the applications on their own and do it with any data containing geospatial information. Also, it is very likely that he will be given similar task in the near future, but with a different data type and the developer would like to re-use as much work as possible from the first time.

\textit{Contribution:} Thanks to our \emph{application generator}, the developer will have to focus only on developing the actual visualizer plugin. The generator will work both as a \emph{framework} and a \emph{platform}, solving many problems up front. As as a \emph{platform}, the generator will handle user agenda, extendability, application management (creating, updating, deleting, publishing), common application configuration and so on. As a \emph{framework}, the generator will define a standard  interface which will allow seamless integration of new plugins. Plus it will provide many drop-in solutions for typical challenges that the developer might encounter across multiple plugins (e.g. caching, multi-language support, support for fixing missing or incorrect data labels, re-usable user interface components etc.).

\item \textbf{Data Analyst}

\textit{Description:} A data analyst is not necessarily a developer, but he has enough knowledge to work with data on a lower level, typically convert data to RDF format or to augment existing RDF data sets with different meta-data. 

\textit{Motivation/Use case:} Let us say that a data analyst works at the same company as our developer. He has a couple of interesting data sets that he would like to make available for other employees so that they could use them to generate applications with the plugin developed by the developer.

\textit{Contribution:} By implementing the interface defined by the \emph{framework}, each visualizer plugin describes what the input data should look like. Therefore the data analyst has clear instructions on how to prepare a data set to make it compatible with the visualizer plugin. After preparing such data set, the data analyst will be able to upload it to the generator and make it available for all other users that can use the data set for generating new applications.

\item \textbf{User}

\textit{Description: }This is the non-developer with no technical knowledge. He does possess at least some basic understanding of the data though.

\textit{Motivation/Use case: }A user wants to create a new interactive application from a data set and make it available online. For example, he might want to create a map application from the data set with geospatial information, but would be interested only in certain type of data from a certain area.

\textit{Contribution: }The \emph{application generator} will let this user leverage the work that has been already done by the developer and the data analyst. He will be able to select a prepared data set (or choose his own), the generator will analyze the content of the data set and based on the results, the generator will offer the user a list of visualization plugins that are compatible with the data set. By choosing one plugin, the user will proceed to a configuration phase where he will be allowed to create the application according to his needs. The configuration possibilities will differ depending on the selected visualizer plugin, but typically the user will be allowed filter the data (select a subset) and to tune the level of interactivity for the audience. For example, he could either create a completely static visualization with pre-filtered data, or he could let the audience decide what they want to see. This will greatly increase the re-usability of data sets and visualizer plugins. A single data source with a single visualization plugin will work as a potential source for many applications, each serving a different purpose. None of these applications will involve any more work from the developer or from the data analyst.

Before publishing the application, the user will be allowed to fill missing information (e.g. missing labels in target language) and provide basic application meta data (name and description). After publishing, the application will be available online from a unique URL.

\item \textbf{Audience}

\textit{Description:} The end-user with no knowledge at all, just some basic computer skills.

\textit{Motivation/Use case:} Audience wants to use the application and wants to understand it.

\textit{Contribution:} The \emph{application generator} will greatly simplify the process of creating new applications which will just by itself result in an increased number of generated applications. Therefore the audience will benefit just from the bare existence of the generator because without the generator, there would be way less applications and the data potential would be wasted. Plus thanks to the configuration phase, the applications will not be just raw visualizations and they will offer better comfort and user experience for the audience.

\end{itemize}

\section*{Goal of this thesis}

The goal of this thesis will be to create a \emph{Linked Data based application generator} with all the features listed in the previous section. The application generator will work as a \textit{framework}, easily extendable with new visualization plugins that will allow generating applications from different types of data. It will define a clear interface that the new plugins will have to implement. Also many documented drop-in solutions for typical problems and situations will be available for the developer.

The application generator will also be a \textit{platform}, providing space for developers to register new plugins, space for data analyst to publish new data sets and space for common users to generate, manage and publish their applications.

Besides the generator itself, we will also implement two visualizer plugins to demonstrate the abilities of the \textit{framework} and the \textit{platform}.

\section*{Structure of the text}

In Chapter 1, we briefly describe RDF and Linked Data and provide related information necessary for understanding the rest of the thesis. In Chapter 2, we focus on similar existing tools and compare them to our proposed generator. In Chapter 3, we examine in depth the tool LinkedPipes as we will base our generator on top of it.  In Chapter 4, we look more in detail into the required features and we present steps and decisions necessary for achieving our goals. In this chapter, we also analyze and defend our decision to use LinkedPipes as a base for our generator, plus we describe how it has to be modified to fit our intentions. In Chapter 5, we dive into implementation details. Especially we focus on the framework aspect. We show how a new visualizer plugin can be integrated into the generator and what tools are available for the developer. In Chapter 6, we describe in detail the two visualizer plugins, D3.js Chord Visualizer and Google Maps Visualizer, that we implemented to demonstrate our generator's abilities. Related to the D3.js Chord Visualizer, there is a short theoretical part in this chapter that deals with representing and sampling graph structures (actual graphs as understood in graph theory) using RDF and SPARQL queries. In the final Chapter 7, we describe possible future work.