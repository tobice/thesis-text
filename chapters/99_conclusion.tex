\chapter*{Conclusion}
\addcontentsline{toc}{chapter}{Conclusion}

We implemented the proposed \emph{application generator} which enables users to generate interactive web applications from RDF data. The user starts by selecting a source of RDF data. The \emph{application generator} analyzes the data and offers a list of \emph{visualizers} that are able to visualize the data. The user selects the \emph{visualizer} he is interested in and uses it to generate a new application. The \emph{application generator} allows the user to configure the application and then publish it.

We examined several existing related tools (Chapter~\ref{chap:related-work}) which helped us to specify the list of features (Section~\ref{sec:rw:features}) that we decided our \emph{application generator} should support. This also showed that compared to other data representation formats, Linked Data offer the biggest potential for automatic analysis and automatic application generation.

Our \emph{application generator} is integrated (Section~\ref{sec:system_proposal:integration}) into LinkedPipes Visualization (Chapter~\ref{chap:linkedpipes-visualization}). It utilizes its underlying LDVM implementation for fully automated and easily extendable Linked Data analysis. We extended LinkedPipes Visualization with all the necessary functionality required by our \emph{application generator}. Specifically, we added support for two new RDF vocabularies, RGML and FRESNEL.

The biggest asset of \emph{application generator} is the \emph{configuration phase} (Section~\ref{sec:system_proposal:contribution}). Each \emph{visualizer} provides a \emph{configurator} user interface unique to the type of data the \emph{visualizer} supports. The \emph{configurator} interface enables the user to configure the application before it gets published. The configuration possibilities differ depending on the \emph{visualizer}, but typically the user is allowed to filter the data and tune the level of the application interactivity for the audience. It is also possible for example to provide missing text labels for individual entities in the data set.

Our \emph{application generator} is \emph{non-developers friendly}. The whole process of generating a new application (Section~\ref{sec:implementation:use-case-scenario}) requires zero technical knowledge and anyone with the base computer skills is able to generate a new application.

Our \emph{application generator} works as a \emph{platform} (Section~\ref{sec:implementation:overview}). Every user has its own account which he uses to manage his published application. The \emph{application generator} contains a catalog of published applications and also a simple repository of available data source that any user can utilize to generate applications.

Our \emph{application generator} is a \emph{framework}. Firstly, the \emph{framework} defines a clear way how new \emph{visualizer} can be seamlessly integrated into the \emph{application generator} which makes it easily extendable. We explained the process of integration a new \emph{visualizer} in a step-by-step guide (Section~\ref{sec:implementation:integrating-visualizer}) which is a part of this thesis. Secondly, the \emph{framework} provides the developer with several read-to-use drop-in solutions (Section~\ref{sec:implementation:advanced-features}) for some common tasks that can significantly speed up the process of developing a new \emph{visualizer}. Those common tasks for example include saving and loading the application configuration (Section~\ref{sec:implementation:integrating-visualizer:configuration}) or multiple language support.

To showcase the capabilities of the \emph{platform} and the \emph{framework}, we developed two \emph{visualizers}. The first one is D3.js Chord Visualizer (Section~\ref{sec:visualizers:chord}) which visualizes graph data represented in RDF with a chord diagram. While working on this \emph{visualizer}, we implemented and analyzed algorithms for fulltext search, for calculating a graph adjacency matrix and for graph sampling, all of which over a graph represented in RDF (Chapter~\ref{chap:theoretical-attachment}). The second one is Google Maps Visualizer (Section~\ref{sec:visualizers:google-maps}) which visualizes geospatial data on a map. This \emph{visualizer} already existed in LinkedPipes Visualization and we decided to re-implement it for our \emph{application generation} to show the benefits of the \emph{application generator} concept over LinkedPipes Visualization (Section~\ref{sec:visualizers:google-maps:advantages}).

A major part of our work was developing the frontend of the \emph{application generator}. Looking back, we feel ambivalent about the tools (the development stack) that we chose (Section~\ref{sec:implementation:frontend-development-stack}). On one hand, many of the principles and approaches that these tools brought in proved a significant potential (e.g. Section~\ref{sec:implementation:advanced-features:custom-labels-editor} or the whole integration process on the frontend side Section~\ref{sec:implementation:integrating-visualizer}). On the other hand, we spent lots of time dealing with rather low-level unrelated implementation problems that have been solved many times before but not for this particular set of tools, at least not to the extent that we could simply follow existing patterns and recommendations (Sections~\ref{sec:implementation:frontend-conventions} and~\ref{sec:implementation:advanced-features:miscellaneous}). It is possible that if we had chosen a perhaps older but more established frontend framework, we would have had more time for the actual work on the \emph{application generator}.

What might be surprising to the reader is that although we developed a \emph{Linked Data driven application generator}, we are not able to generate any applications from most of the available Linked Data data. If you at this moment use DBpedia~\cite{dbpedia} as the input source (and DBPedia is a really huge source), the internal analysis will yield no results. On the contrary, if you take any of the tabular data based generators that we described in related work (Chapter~\ref{chap:related-work}), such as Miga Data Viewer, those tools will generate an application from any valid source of tabular data. Does this mean that we failed to deliver what we promised?

Clearly, our \emph{application generator} is only as powerful as its \emph{visualizers}. It is important to note that unlike the aforementioned tabular data, the RDF data are very versatile in what kind of information they can carry. That makes it very hard (possibly impossible) to create an almighty universal \emph{visualizer} capable of visualizing any RDF data (unless the \emph{visualizer} is very simplistic). However, we are confident that given the way the \emph{application generator} is designed, being both a \emph{platform} and a \emph{framework}, it lays a solid ground for future potential extensions. We believe we fulfilled the goal of this thesis.
