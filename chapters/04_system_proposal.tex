\chapter{System proposal}

Based on the information we gathered in the previous chapters, we will now propose our own \emph{application generator}. We will give a detailed description and justification of the characteristics and features that our system will have. 

At the beginning of the chapter covering Related Work (Section \ref{sec:rw:definition}), we attempted to define what a \emph{data-driven application generator} is. We admitted that this definition is rather vague and subjective but it is important because it gives as a starting point for making our own tool of this kind. We will now briefly repeat that definition.

A \emph{data-driven application generator} is a tool that takes an input data set provided by the user and produces an \emph{application} based on that data set. Defining an \emph{application} is rather difficult. For us, the main aspect is \emph{interactivity}, i.e., the generated \emph{application} will let the user interact in some way with the source data set. Moreover, a generated \emph{application} should be persistent and to an extent independent on the \emph{generator}. When the \emph{generator} is closed, the \emph{application} should not seize to exist.

\section{Proposed features}

In Section \ref{sec:rw:features}, we presented a list features that we focused on when examining individual related tools. This list of features allowed us to compare the tools between each other in a more organized way. This list also contains the exact features that we would like our tool to have. Let us walk through that list again and explain why we think our \emph{application generator} should support these features.

\begin{itemize}
\item \emph{Linked Data support}. The majority of the examined tools supported only some kind of tabular data. That proved itself to be a very limiting factor. Each tool required the user to somehow provide the missing \emph{semantic} information (e.g. by describing the data using a \emph{schema}) and even then the tool either generated applications that looked all more or less the same (Miga Data Viewer, Citadel on the Move), or required the user to manually build the whole application (Tableau, Avelca, Exhibit). Linked Data carry way more \emph{semantic} information with them and can be \emph{understood} by the \emph{application generator}. That means that the process of generating applications can be automated (the user does not have to explain to the \emph{generator} what the input data mean) and also as the Linked Data can express way more kinds of information, way more types of applications can be generated from that data.
% * <tobiaspotocek@gmail.com> 2016-06-17T18:59:01.462Z:
%
% > Miga Data Viewer
%
% Do I need to make a reference back to the Related Work chapter whenever I mention one of the tools?
%
% ^.
\item \emph{Extendability}. All applications generated for example by Miga Data Viewer looked more or less the same. As explained in the previous paragraph, that was caused to a large extent by the characteristics of the tabular data. Linked Data, on the other hand, are way more versatile. Plus we can hardly imagine that we could come up with a universal solution that would work for any type of data. Therefore making the \emph{generator} extendable is a necessity.
\item \emph{Data analysis}. A \emph{generator} that automatically analyzes the input data and makes decisions based on how it \emph{understands} the data, is definitely a more capable \emph{generator}. This is directly related to the \emph{Linked Data support}. It is the Linked Data that allow such analysis possible (we are not claiming that it is not possible to run some kind of smart analysis on tabular data, but as it would involve lots of guessing, the results might be debatable). Just to repeat what we have already said: such a smart \emph{generator} can significantly speed up the process of application generation by making it automated (to an extent) and it can support significantly more types of information with significantly more types of applications.
\item \emph{Online sharing}. Being able to \emph{share} the final generated application with others is the purpose of all these efforts. We aim to use the whole potential of the data around us and keeping our findings just to ourselves would clearly waste that potential.
\item \emph{Non-developers friendly}. Miga Data Viewer or Exhibit proved that even tools that require their users to have some programming skills, can help a lot by reducing the amount of work necessary for generating an application. Also these tools typically offered a high level of flexibility. Nevertheless, most of our potential users have zero developer skills and we would like to allow these users to generate applications using our \emph{generator}.
\item \emph{Platform}. Tools in the form of an online platform (Avelca, Tableau, Citadel on the Move, Payola) clearly make the whole agenda around applications (creating, managing, sharing) simpler for the user. We want our tool to work as a platform as well.
\item \emph{Configuration}. We have seen two extremes among the examined tools. LinkedPipes Visualization simply produced the visualization and gave user no possibility to configure it before publishing. Tools like Tableau, Avelca or Exhibit, on the other hand, represented the other extreme. The user had to actually \emph{build} the whole application from scratch. Miga Data Viewer was somewhere in between. Most of the application was automatically generated but the user was still allowed to change certain aspects. This is the way that we would like to choose for our \emph{application generator} as well. We want to find a compromise solution that would allow the user to quickly generate new applications and yet it would still give him some space to influence how the application should look like before it gets published.
\end{itemize}

\section{Advantages and disadvantages of integration into LinkedPipes Visualization}

We proposed that some kind of \emph{data analysis} should be a part of our \emph{application generator}. Automatic analysis of Linked Data, however, is a vast topic. Within the scope of this thesis, we would be probably able to come up only with a very basic solution consisting of simple rules such as  \textit{"This entity is an address or GPS coordinates, let us display it on a map."} or \textit{"These are some statistical data, let us visualize it using a graph"}. We can say that Miga Data Viewer works this way to an extent.

We believe that it is not always necessary to reinvent what has been already invented and that we would rather give ourselves a head start by re-using an existing solution. For various reasons, we decided to integrate our \emph{application generator} into LinkedPipes Visualization. Let us now walk through those reasons.

\begin{enumerate}
\item Through LinkedPipes Visualization we get a strong analytical and visualization framework that we can immediately use. We will utilize the \emph{discovery} algorithm which will automatically tell use how the input data can be visualized, i.e., what kinds of applications can be generated.
\item We will greatly benefit from the LinkedPipes Visualization implementation of LVDM. Firstly, it will make our generator easily extendable to support new types of data through new LDVM components. Secondly, any LDVM component following the format defined by this implementation will automatically work in our generator as well (with the limitations explained in Section \ref{sec:linkedpipes:component_registration}).
\item On a programming level, LinkedPipes Visualization already contains lots of ready-to-use solutions for working with RDF data (querying Virtuoso triplestore, converting RDF to JSON etc.). It also offers a programmatical API providing access to the \emph{discovery} algorithm (launching, accessing results etc).
\item Finally, we admit that what played an important role in our decision process was the fact that we had direct personal access to the authors of LinkedPipes Visualization. That significantly sped up our work.

\end{enumerate}
This decision has also disadvantages. The original guidelines for this thesis suggest that the user should be able to combine together different views for different types of data. Such views should be possible to display in a selected predefined layout before publishing the app. We have seen this approach for example in Tableau. However, LinkedPipes Visualization visualizers use a completely different approach as they are typically very domain specific. Each visualizer focuses on a single type of data, for example map data, and then offers a rich (but static) user interface which is specifically designed to work with this particular type of data (for example, it displays controls allowing the user to filter the visualized data). This is a direct consequence of the underlying LDVM pipeline \emph{discovery} algorithm.

Unfortunately, if we are to build on top LinkedPipes Visualization, our applications (which will directly correspond to the \emph{visualizers}) cannot look and work much differently. Nevertheless, both approaches are viable and both have their advantages and disadvantages. By adapting the approach of domain-specific applications, we will reduce the application configurability and it will not be possible to generate dashboard-like applications known from Tableau. On the other hand, our domain-specific applications will offer richer user interfaces allowing more advanced work with the visualized data.

\section{Contribution}

\textit{What makes our generator better than LinkedPipes Visualization}

Let us describe more in depth how our \emph{application generator} will work and how it will help its users. To make it more organized, we will separate the descriptions by different type of users that should benefit from our generator. For each type of user, we will describe his role, his motivation (or perhaps a use case typical for this user) and eventually the contribution of this generator, i. e., how this generator is going to help him.

\begin{itemize}
\item \textbf{Developer}

\textit{Description:} A software engineer with deep technical knowledge.

\textit{Motivation/Use case:} A developer is given a task at his job to create an interactive online map application that visualizes provided company geospatial data. The developer is smart and wants to avoid doing the same thing again the next time. Therefore, he would prefer to create a universal solution that would allow other non-developer employees to create the applications on their own and do it with any data containing geospatial information. Also, it is very likely that he will be given similar task in the near future, but with a different data type and the developer would like to re-use as much work as possible from the first time.

\textit{Contribution:} Thanks to our \emph{application generator}, the developer will have to focus only on developing the actual visualizer plugin. The generator will work both as a \textit{framework} and a \textit{platform}, solving many problems up front. As as a \textit{platform}, the generator will handle user agenda, extendability, application management (creating, updating, deleting, publishing), common application configuration and so on. As a \textit{framework}, the generator will define a standard  interface which will allow seamless integration of new plugins. Plus it will provide many drop-in solutions for typical challenges that the developer might encounter across multiple plugins (e.g. caching, multi-language support, support for fixing missing or incorrect data labels, re-usable user interface components etc.).

\item \textbf{Data Analyst}

\textit{Description:} A data analyst is not necessarily a developer, but he has enough knowledge to work with data on a lower level, typically convert data to RDF format or to augment existing RDF data sets with different meta-data. 

\textit{Motivation/Use case:} Let us say that a data analyst works at the same company as our developer. He has a couple of interesting data sets that he would like to make available for other employees so that they could use them to generate applications with the plugin developed by the developer.

\textit{Contribution:} By implementing the interface defined by the framework, each visualizer plugin describes what the input data should look like. Therefore the data analyst has clear instructions on how to prepare a data set to make it compatible with the visualizer plugin. After preparing such data set, the data analyst will be able to upload it to the generator and make it available for all other users that can use the data set for generating new applications.

\item \textbf{User}

\textit{Description: }This is the non-developer with no technical knowledge. He does possess at least some basic understanding of the data though.

\textit{Motivation/Use case: }A user wants to create a new interactive application from a data set and make it available online. For example, he might want to create a map application from the data set with geospatial information, but would be interested only in certain type of data from a certain area.

\textit{Contribution: }The application generator will let this user leverage the work that has been already done by the developer and the data analyst. He will be able to select a prepared data set (or choose his own), the generator will analyze the content of the data set and based on the results, the generator will offer the user a list of visualization plugins that are compatible with the data set. By choosing one plugin, the user will proceed to a configuration phase where he will be allowed to create the application according to his needs. The configuration possibilities will differ depending on the selected visualizer plugin, but typically the user will be allowed filter the data (select a subset) and to tune the level of interactivity for the audience. For example, he could either create a completely static visualization with pre-filtered data, or he could let the audience decide what they want to see. This will greatly increase the re-usability of data sets and visualizer plugins. A single data source with a single visualization plugin will work as a potential source for many applications, each serving a different purpose. None of these applications will involve any more work from the developer or from the data analyst.

Before publishing the application, the user will be allowed to fill missing information (e.g. missing labels in target language) and provide basic application meta data (name and description). After publishing, the application will be available online from a unique URL.

\item \textbf{Audience}

\textit{Description:} The end-user with no knowledge at all, just some basic computer skills.

\textit{Motivation/Use case:} Audience wants to use the application and wants to understand it.

\textit{Contribution:} The application generator will greatly simplify the process of creating new applications which will just by itself result in an increased number of generated applications. Therefore the audience will benefit just from the bare existence of the generator because without the generator, there would be way less applications and the data potential would be wasted. Plus thanks to the configuration phase, the applications will not be just raw visualizations and they will offer better comfort and user experience for the audience.
\end{itemize}

\textit{Sum up to contribution}

\section{Design overview}

\textit{Perhaps last part: how we want to achieve all of this. Describe how we are going to extend LinkedPipes. Replace the UI, visualizer LDVM component -> application template, configurator vs application. Extend for user. Mock-ups But that might rather be part of the implementation part.}	


"Once the generator is closed, the application should disappear"

\section{Visualizers}

\textit{Proposed visualizers}


	